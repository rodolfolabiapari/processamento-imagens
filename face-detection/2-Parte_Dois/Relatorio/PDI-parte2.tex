% !TeX spellcheck = pt_BR
\documentclass[12pt]{article}

\usepackage{sbc-template}

\usepackage{graphicx,url}

\usepackage[brazil]{babel}
\usepackage{amsmath}
\usepackage{todonotes}
\usepackage[utf8]{inputenc}

     
\sloppy

\title{Disciplina de Processamento Digital de Imagem \\ Trabalho 2 - Referencial Bibliográfico sobre \textit{Face Detection}}

\author{Rodolfo Labiapari Mansur Guimarães}


\address{Departamento de Computação -- Universidade Federal de Ouro Preto
  (UFOP)\\
  34.000-000 -- Ouro Preto -- OP -- Brasil
  \email{rodolfolabiapari@decom.ufop.br}
}

\begin{document} 

\maketitle

\begin{resumo} 
  Deverá ser feita uma revisão crítica de pelo menos 5 artigos, descrevendo de forma geral a proposta apresentada nos diferentes artigos. 
  O relatório deve conter (no máximo uma ou duas páginas)
  a) Um resumo breve do artigo, o qual deve conter uma visão geral do que foi feito, quais foram os métodos utilizados, assim como os respectivos resultados
  b) Uma crítica do artigo, ou seja, deve indicar quais problemas não foram endereçados na proposta apresentada pelos autores, assim como os méritos do trabalho.
\end{resumo}

\section{Introdução}
	Detecção de face é um problema de visão computacional. Métodos modernos podem facilmente detectar faces em ambientes controlados, ou seja, com fundo respeitando um certo padrão, sem obstrução facial como óculos, barba, etc. 
	
	Como o estado da arte bem concreto em detecção de face em ambientes controlados, as pesquisas mas recentemente começaram a buscar métodos que tivessem foco principalmente em detecção de face em ambientes onde existe uma complexidade maior. As pesquisas em detecção de face hoje procura métodos eficientes e rápidos para detecção de faces onde elas aparecem não só fora do perfil mas sim, com distâncias variadas, oclusões na face entre outras possibilidades que poderia representar mais ainda situações reais.
	
	
	
Modern face detectors can easily detect near frontal
faces. Recent research in this area focuses more on the
uncontrolled face detection problem, where a number of
factors such as pose changes, exaggerated expressions and
extreme illuminations can lead to large visual variations in
face appearance, and can severely degrade the robustness of
the face detector.
The difficulties in face detection mainly come from two
aspects: 1) the large visual variations of human faces in the
cluttered backgrounds; 2) the large search space of possible
face positions and face sizes. The former one requires
\subsection{ela deve introducir ao leitor ao tema do estudo e, fundamentalmente, dar ao leitor as razões que justificam o objetivo do estudo.}
\subsection{a descrição do problema, a justificativa, assim como a uma breve revisão bibliográfica.}
\subsection{a solução proposta deve ser descrita brevemente, resaltando as diferenças e vantagens frente aos métodos da literatura}
\subsection{o último parrafo deve conter um resumo das seções do artigo (escrever esse párrafo no só na versão final).}

\section{Desenvolvimento}
Desenvolvimento: nesta seção é desenvolvido o modelo proposto, ele deve ser explicado em detalhes
\subsection{Deve conter a fundamentação teórica}
\subsection{Use figuras para explicar o funcionamento}
\subsection{Comece pelo modelo mais genérico e desenvolva cada parte em detalhes nas sub-seções posteriores}


\bibliographystyle{sbc}
\bibliography{sbc-template}

\end{document}
