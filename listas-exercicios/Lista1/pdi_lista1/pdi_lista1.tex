\documentclass[12pt]{article}

\usepackage{sbc-template}

\usepackage{graphicx,url}

\usepackage[brazil]{babel}   
%\usepackage[latin1]{inputenc}  
\usepackage[utf8]{inputenc}  

\usepackage{minted}
     
\sloppy

\title{Processamento de Imagens \\ Lista 1}

\author{Rodolfo Labiapari Mansur Guimarães}


\address{Departamento de Computação -- Universidade Federal de Ouro Preto (UFOP)\\
  \email{rodolfolabiapari@gmail.com}
}

\begin{document} 

\maketitle

\section{Exercício 1}

\begin{minted} [frame=lines, framesep=2mm, tabsize=3, breaklines=true, baselinestretch=1.2, linenos, fontsize=\footnotesize ]{octave}
	Exe1 = [1:7; 9:-2:-3; 2.^[2:8]];
\end{minted}



\section{Exercício 2}
	A)
	\begin{minted} [frame=lines, framesep=2mm, tabsize=3, breaklines=true, baselinestretch=1.2, linenos, fontsize=\footnotesize ]{octave}
		
	vet1 = [1:5];
	vet2 = [1 1 1];
	
	resultado = (vet2') * vet1;
	\end{minted}
	B)
	\begin{minted} [frame=lines, framesep=2mm, tabsize=3, breaklines=true, baselinestretch=1.2, linenos, fontsize=\footnotesize ]{octave}
	vet1 = [0:4];
	vet2 = [1 1 1];
	
	resultado = (vet1') * vet2;
	\end{minted}
	
\section{Exercício 3}
	
	\begin{minted} [frame=lines, framesep=2mm, tabsize=3, breaklines=true, baselinestretch=1.2, linenos, fontsize=\footnotesize ]{octave}
	
	vet1 = [1:1.5:12];
	somatorio = sum(vet1);
	
	A3 = ln(2 + somatorio + somatorio ^ 2);
	
	B3 = exp(somatorio * (1 + cos(3 * somatorio)));
	
	C3 = cos(somatorio).^2 + sin(somatorio).^2;
	\end{minted}
	
\section{Exercício 4}
	
	\begin{minted} [frame=lines, framesep=2mm, tabsize=3, breaklines=true, baselinestretch=1.2, linenos, fontsize=\footnotesize ]{octave}
		
	intervalo = [0.01:0.01:0.1];
	
	plot(intervalo, sin(1./intervalo));
	print -dpng some_name.png;
	\end{minted}
	
	{\footnotesize
	\begin{verbatim}
	+-----------------------------------------------------------------------------+
	|                                                                             |
	|   1 ++------------+-------------+------------+-------------+------------++  |
	|     +-------------------------------------------------------------------++  |
	|     |                   | |           ||           | |                  ||  |
	|     |                   | |           | |          |  |                 ||  |
	|     |                  |   |         |  |         |    |                ||  |
	| 0.5 |+                |     |       |    |        |     |               |+  |
	|     |                 |      |      |    |        |     |               ||  |
	|     |                |       |     |     |       |       |              ||  |
	|     |                |        |   |       |      |        |             ||  |
	|   0 |+              |          | |        |     |         |             |+  |
	|     |              |           | |        |     |          +            ||  |
	|     |              |            +          |   |            |           ||  |
	|     |            -+                        |   |             |          ||  |
	|     |        ----                           |  |             |          ||  |
	|-0.5 |+     +-                               | |               |         |+  |
	|     |                                       | |                |        ||  |
	|     |                                        |                  |      -||  |
	|     |                                        +                  |    -- ||  |
	|     |             +             +            +             +     | --   |+  |
	|  -1 +-------------------------------------------------------------------++  |
	|     0           0.02          0.04         0.06          0.08           0.1 |
	+-----------------------------------------------------------------------------+
	
	\end{verbatim}
	}
	
\section{Exercício 5}
	
	\begin{minted} [frame=lines, framesep=2mm, tabsize=3, breaklines=true, baselinestretch=1.2, linenos, fontsize=\footnotesize ]{octave}
	Notas = [7.5 8.0 9.0 ; 6.7 7.7 5.4 ; 8.0 9.2 7.4 ; 6.6 6.6 6.6 ; 5.0 8.0 7.0]
	
	A  = (sum([Notas Notas(:,3)], 2)) ./ 4
	
	B1 = mean(Notas)
	B2 = mean((sum([Notas Notas(:,3)], 2)) ./ 4)
	\end{minted}
	
\section{Exercício 6}
	
	Em anexo.
	
\section{Exercício }
	
	\begin{minted} [frame=lines, framesep=2mm, tabsize=3, breaklines=true, baselinestretch=1.2, linenos, fontsize=\footnotesize ]{octave}
		
	vetor = [1:3:300];
	
	media_geometrica  = ((prod(vetor)) ^ (1 / length(vetor)))
	media_aritimetica = sum(vetor) / size(vetor, 2)
	\end{minted}
\end{document}
